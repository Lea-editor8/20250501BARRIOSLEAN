\documentclass[oneside,onecolumn,openany,final]{book}

% Deben cargarse temprano para parches de macros
\usepackage{etoolbox}
\usepackage{ifthen}
\usepackage{xparse}

% Soporte para fuentes OpenType
\usepackage{fontspec}

\usepackage{polyglossia}
\setmainlanguage{spanish}
\usepackage{csquotes}

% paquetes varios
\usepackage{calc}
\usepackage{nccfoots}
\usepackage{booktabs}
\usepackage{rotating}
\usepackage{graphicx}
\usepackage{svg}
\usepackage[final]{pdfpages}
\usepackage{caption}
\usepackage{ragged2e}
\usepackage{xcolor}
\usepackage[framemethod=tikz]{mdframed}
\usepackage{bchart}
\usepackage[most]{tcolorbox}

% diseño de listas (opcion 1) con paralist
\usepackage{paralist}
\setdefaultenum{1)}{a)}{i)}{}
\pltopsep=0.5mm
\plitemsep=0mm

% diseño de listas (opcion 2) con enumitem
\usepackage{enumitem}
\setlist{nosep,topsep=4pt}

% manejo de índices
\usepackage{imakeidx}
\makeindex[name=names,title=Índice de autores]
\makeindex
\makeindex[name=foo,title=Índice de conceptos]
\makeindex[name=foo3,title=Índice onomástico]

% manejo de glosarios
\usepackage[acronym,sanitizesort]{glossaries}
\makenoidxglossaries

% volcamos las notas como capítulo separado
\usepackage{endnotes,chngcntr}
\def\notesname{Notas}
\let\footnote=\endnote
\makeatletter
\renewcommand\enoteheading{%
    \setcounter{secnumdepth}{-2}
    \chapter*{\notesname\markboth{}{}}
    \mbox{}\par\vskip-\baselineskip
    \let\@afterindentfalse\@afterindenttrue
}
\makeatother
\let\latexchapter\chapter
\RenewDocumentCommand{\chapter}{som}{%
    \IfBooleanTF{#1}
    {\latexchapter*{#3}}
    {\IfNoValueTF{#2}
        {\latexchapter{#3}}
        {\latexchapter[#2]{#3}}%
        \addtoendnotes{%
            \noexpand\enotedivision{\noexpand\subsection}
            {\thechapter \, \unexpanded{#3}}}% \chaptername\
    }%
}
\makeatletter
\def\enotedivision#1#2{\@ifnextchar\enotedivision{}{#1{#2}}}
\makeatletter

% elegimos el estandar para las referencias
\def\estandar{veronaC}
%\def\estandar{veronaM}
%\def\estandar{VerboseIbid}
%\def\estandar{APA}
%\def\estandar{ISO690}
%\def\estandar{numeric}
%\def\estandar{custom}
\input{./files/biblatex-\estandar-config.tex}

% configuración de los valores decimales de los cuadros
\usepackage{siunitx}
\sisetup{output-decimal-marker={.},
	group-separator={\hspace{0.15em}},
	group-minimum-digits=3,
	table-text-alignment=center,
	detect-all,
	per-mode=fraction}

% aseguramos igualdad al valor de la raya del medio
\newcommand{\rdm}[1]{--#1--}
\newcommand{\rdmq}[2]{--#1--\penalty10000 #2}

% función para generar una raya de unión de palabras que habilita el corte de palabra
% se usa entre llaves {\rdc}
\makeatletter
\def\thinskip{\hskip 0.01em\relax}
\def\rdc{\thinskip--\thinskip}
\makeatother

% condicionales para los formatos de salida
\newif\ifPDF
\newif\ifBNPDF
\newif\ifPNGEPUB
\newif\ifHTMLEPUB
\newif\ifHTML
\newif\ifXML

% fin del archivo de configuración
