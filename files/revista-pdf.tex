%  archivo cabecera para revistas con LuaLaTeX

% Deben cargarse temprano para parches de macros
\usepackage{etoolbox}
\usepackage{ifthen}

% Paquetes báscos
\usepackage{graphicx}             % Manejo de imágenes
\usepackage{xcolor}               % Soporte para colores
\usepackage{booktabs}             % Tablas de calidad
\usepackage{multirow}             % Celdas que abarcan múltiples filas
\usepackage{mathtools}            % Extensión de amsmath
\usepackage{amsfonts,amssymb}     % Fuentes y símbolos matemáticos
\usepackage{fancyhdr}             % Encabezados y pies de página avanzados
\usepackage{titlesec}             % Personalización de títulos de sección
\usepackage{abstract}             % Personalización de resúmenes
\usepackage{authblk}              % Manejo avanzado de autores y afiliaciones
\usepackage{tabularx}             % Tablas avanzadas
\usepackage{mdframed}             % Marcos y cajas
\usepackage{float}                % Control de posicionamiento de objetos flotantes
\usepackage{caption}              % Personalización de leyendas
\usepackage{subcaption}           % Subfiguras
\usepackage{enumitem}             % Listas personalizadas
\usepackage{fontawesome5}         % Iconos
\usepackage{doi}                  % Formateo de DOIs
\usepackage{orcidlink}            % Soporte para ORCID
\usepackage{lastpage}             % Referencia a la última página
\usepackage{textcomp}             % Símbolos de texto adicionales
\usepackage{csquotes}             % Comillas contextuales
\usepackage{siunitx}              % Unidades del SI
\usepackage{marginnote}           % Notas al margen
\usepackage{pdfpages}             % Inclusión de páginas PDF

\usepackage{fontspec}% Soporte para fuentes OpenType
\usepackage[final]{microtype}% Mejoras tipográficas micro
\microtypecontext{spacing=nonfrench}% Mejora espaciado sin modificar el interletrado
% solo trabajamos con protusion y sin expansion para Libertinus
\SetProtrusion
[ name=libertinus ]
{ encoding = * }
{ . = {50,50}, , = {40,40}, - = {30,30}, " = {50,50}, ( = {40,50}, ) = {50,40} }
\renewcommand{\normalsize}{\fontsize{10pt}{14pt}\selectfont}
\topskip=14pt

% Fuente principal Libertinus con características tipográficas
\setmainfont{Libertinus Serif}
[Numbers={OldStyle,Proportional},
Ligatures={TeX,Common,Discretionary},
RawFeature={+cv01,+cv02}, % Variantes de caracteres opcionales
Scale=1.18]

% Matemáticas con Libertinus Math
\usepackage{unicode-math}
\setmathfont{Libertinus Math}[Scale=MatchLowercase]

%% Configuración de fuentes para chino
% \usepackage{luatexja-fontspec}
% \setmainjfont{FandolSong}

\setsansfont[Scale=MatchLowercase,
Ligatures=TeX,
Extension=.otf,
UprightFont=*-Regular,
ItalicFont=*-Italic,
BoldFont=*-SemiBold,
BoldItalicFont=*-SemiBoldItalic]{IBMPlexSansCondensed}

\setmonofont[Scale=0.91,
Extension=.otf,
UprightFont=*-Regular,
ItalicFont = IBMPlexMono-Italic.otf,
BoldFont = IBMPlexMono-Bold.otf,
BoldItalicFont = IBMPlexMono-BoldItalic.otf]{IBMPlexMono.otf}

% Referencias
\usepackage[style=apa,backend=biber]{biblatex}
\DeclareLanguageMapping{spanish}{spanish-apa}% no existe, lo tengo en desarrollo

% Configuración de idioma
\usepackage[french,portuguese,italian,english,german,spanish,es-ucroman,es-noshorthands]{babel}
% Citas automáticas con estilo idiomático correcto
\usepackage[autostyle=true]{csquotes}
\frenchspacing% Espaciado uniforme después de puntos

% CONFIGURACIÓN DE PÁGINA
\usepackage{geometry}
\geometry{
  a4paper,
  top=2.5cm,
  bottom=2.5cm,
  left=2.5cm,
  right=2.5cm,
  headheight=14pt,
  footskip=1cm
}

% CONFIGURACIÓN DE ENCABEZADOS Y PIES
\pagestyle{fancy}
\fancyhf{}
\renewcommand{\headrulewidth}{0.4pt}
\renewcommand{\footrulewidth}{0.4pt}
\fancyhead[LE,RO]{\thepage}
\fancyhead[LO]{\textit{\journalshortitle}}
\fancyhead[RE]{\textit{\articleheader}}
\fancyfoot[LE,RO]{\textit{\articlefooter}}
\fancyfoot[C]{\textit{\journalwebsite}}

% DEFINICIÓN DEL ENTORNO ARTICLE
\newenvironment{article}{%
  \clearpage
  \phantomsection
  \stepcounter{article}
  \refstepcounter{section}
  \addcontentsline{toc}{section}{\articleauthorsshort{} - \articletitle{}}

  % Encabezado del artículo
  \begingroup
  \centering
  \vspace*{1em}

  % Tipo de artículo
  \textcolor{gray}{\textsc{\articletype}}\\[0.5em]

  % Título
  {\LARGE\bfseries\articletitle\par}
  \ifdefempty{\articlesubtitle}{}{
    \vspace{0.5em}
    {\large\textit{\articlesubtitle}\par}
  }
  \vspace{1.5em}

  % Autores
  {\large\articleauthors\par}
  \vspace{0.5em}

  % Afiliaciones
  {\footnotesize\articleaffiliations\par}
  \vspace{1em}

  % Historial del artículo
  \begin{tabularx}{\textwidth}{>{\raggedright\arraybackslash}X >{\raggedleft\arraybackslash}X}
    \multicolumn{2}{c}{\textbf{\textit{Historial del artículo}}} \\
    Recibido: & \articlereceived \\
    Revisado: & \articlerevised \\
    Aceptado: & \articleaccepted \\
    Publicado en línea: & \articlepublishedonline \\
  \end{tabularx}
  \vspace{0.5em}

  % DOI
  \begin{center}
    \footnotesize DOI: \href{https://doi.org/\articledoi}{\nolinkurl{\articledoi}}
  \end{center}
  \vspace{1em}
  \endgroup

  % Resumen en cuadro
  \begin{mdframed}[linewidth=0.5pt, backgroundcolor=gray!5]
    \begin{center}
      \textbf{RESUMEN}
    \end{center}
    \articleabstract
    \vspace{1em}
    \noindent\textbf{Palabras clave:} \articlekeywords
  \end{mdframed}
  \vspace{1em}

  % Abstract en inglés
  \begin{mdframed}[linewidth=0.5pt, backgroundcolor=gray!5]
    \begin{center}
      \textbf{ABSTRACT}
    \end{center}
    \articleabstracten
    \vspace{1em}
    \noindent\textbf{Keywords:} \articlekeywordsen
  \end{mdframed}
  \vspace{1em}
}{%
  % Sección para contribuciones de autores, financiamiento, etc.
  \clearpage
  \section*{Información adicional}

  \subsection*{Contribuciones de los autores}
  \authorcontributions

  \subsection*{Financiamiento}
  \funding

  \subsection*{Agradecimientos}
  \acknowledgments

  \subsection*{Conflictos de interés}
  \conflictsofinterest

  \subsection*{Disponibilidad de datos}
  \dataavailability

  \subsection*{Material suplementario}
  \supplementarymaterials

  \subsection*{Derechos de autor}
  \copyrightinfo

  % Bibliografía
  \printbibliography[heading=bibintoc, title=Referencias]

  \clearpage
}

% COMANDOS ESPECÍFICOS PARA METADATOS
\newcounter{article}
\newcommand{\articletitle}{}
\newcommand{\articleauthors}{}
\newcommand{\articleauthorsshort}{}
\newcommand{\articleaffiliations}{}
\newcommand{\articleabstract}{}
\newcommand{\articlekeywords}{}
\newcommand{\articledoi}{}
\newcommand{\articleheader}{}
\newcommand{\articlefooter}{}
\newcommand{\articletype}{}
\newcommand{\articlereceived}{}
\newcommand{\articlerevised}{}
\newcommand{\articleaccepted}{}
\newcommand{\articlepublishedonline}{}
\newcommand{\articlesubtitle}{}
\newcommand{\articleabstracten}{}
\newcommand{\articlekeywordsen}{}
\newcommand{\authorcontributions}{}
\newcommand{\funding}{}
\newcommand{\acknowledgments}{}
\newcommand{\conflictsofinterest}{}
\newcommand{\dataavailability}{}
\newcommand{\supplementarymaterials}{}
\newcommand{\copyrightinfo}{}

% CONFIGURACIÓN PARA FIGURAS Y TABLAS
\captionsetup{font=small,labelfont=bf}
\captionsetup[table]{position=top}
\captionsetup[figure]{position=bottom}

% FORMATEO DE SECCIONES
\titleformat{\section}
  {\normalfont\large\bfseries}{\thesection.}{1em}{}
\titleformat{\subsection}
  {\normalfont\normalsize\bfseries}{\thesubsection.}{1em}{}
\titleformat{\subsubsection}
  {\normalfont\normalsize\itshape}{\thesubsubsection.}{1em}{}

% CARGAR COMANDOS PERSONALIZADOS
\input{files/custom-commands.tex}

% Enlaces
\usepackage{hyperref}
\hypersetup{
  colorlinks=true,
  linkcolor=blue,
  filecolor=magenta,
  urlcolor=blue,
  citecolor=green,
  pdftitle={\journaltitle{} Vol. \journalvolume{}, No. \journalissue{}},
  pdfauthor={\journaleditor{}},
  pdfsubject={\journalsubject{}},
  pdfkeywords={\journalkeywords{}}
}
